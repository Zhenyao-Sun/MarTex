\documentclass[12pt,a4paper]{amsart}
\usepackage[margin=1in]{geometry}
\usepackage[hypertexnames=false]{hyperref}
\usepackage{autonum}
\usepackage{mathrsfs}
\usepackage{filecontents}
\newtheorem{thm}{Theorem}[section]
\title[In line title]{ \large Title}
\author{Author's Name}
\address{Author's Name, Author's Adress}
\email{Author's Email}
\thanks{Author's Thanks Info}
\keywords{}
\subjclass[1991]{}
\date{\today}
\begin{document}
\begin{abstract}
Abstract
\end{abstract}
\maketitle
\section{The first section}\label{the-first-section}
\subsection{The first subsection}\label{the-first-subsection}
\subsection{The second subsection}\label{the-second-subsection}
Some text.
Some in line math \(1+1=2.\)
Some display math \[
1+1=2. \label{eq:eq1}
\] Refer the above equation \eqref{eq:eq1}. Cite some result
\cite{examplekey}. Some indents:
\begin{itemize}
\item
  item1
\item
  item2
\end{itemize}
\par
A new paragraph! Display a new Theorem
\begin{thm}\label{thm:thm1}
There is the first theorem.
\end{thm}
\medskip\begin{proof}
There is the proof of Theorem \ref{thm:thm1}.
\end{proof}
\section{The second section}\label{the-second-section}
Enjoy!
\medskip
\bibliographystyle{alpha}
\bibliography{tempbib}
\begin{filecontents}{tempbib.bib}
@book{examplekey,
  title={Title of the article},
  author={Author's Name of the areticle},
  year={2010},
  publisher={Publisher's info}
}
\end{filecontents}
\end{document}